\documentclass[12pt,fleqn]{article}
\usepackage{vkCourseML}
\hypersetup{unicode=true}
%\usepackage[a4paper]{geometry}
\usepackage[hyphenbreaks]{breakurl}

\interfootnotelinepenalty=10000

\begin{document}
\title{Лекция 10\\Градиентный бустинг}
\author{Е.\,А.\,Соколов\\ФКН ВШЭ}
\maketitle

Мы уже разобрались с двумя типами методов построения композиций~--- бустингом и бэггингом,
и познакомились с градиентным бустингом и случайным лесом, которые являются наиболее яркими
представителями этих классов.
На практике реализация градиентного бустинга оказывается очень непростой задачей,
в которой успех зависит от множества тонких моментов.
Мы рассмотрим конкретную реализацию градиентного бустинга~--- пакет XGBoost~\cite{chen16xgboost},
который считается одним из лучших на сегодняшний день.
Это подтверждается, например, активным его использованием в соревнованиях по анализу данных на~\texttt{kaggle.com}.

\section{Extreme Gradient Boosting (XGBoost)}
\subsection{Градиентный бустинг}
Вспомним, что на каждой итерации градиентного бустинга вычисляется вектор сдвигов~$s$,
который показывает, как нужно скорректировать ответы композиции на обучающей выборке,
чтобы как можно сильнее уменьшить ошибку:
\begin{equation}
\label{eq:gbm}
    s
    =
    \left(
        -\left.
        \frac{\partial L}{\partial z}
        \right|_{z = a_{N - 1}(x_i)}
    \right)_{i = 1}^{\ell}
    =
    -\nabla_z
    \sum_{i = 1}^{\ell}
        L(y_i, z_i)
    \Big|_{z_i = a_{N - 1}(x_i)}
\end{equation}
%Вектор сдвигов равен градиенту со знаком минус; будем обозначать этот градиент как~$g$:
%\[
%    g = -s
%\]
После этого новый базовый алгоритм обучается путем минимизации среднеквадратичного отклонения
от вектора сдвигов~$s$:
\[
    b_N(x)
    =
    \argmin_{b \in \AA}
        \sum_{i = 1}^{\ell}
            \left(
                b(x_i) - s_i
            \right)^2
\]
%Также мы обсуждали возможность вычисления шага с помощью методов второго порядка.
%В этом случае задача поиска базового алгоритма примет вид
%\begin{equation}
%\label{eq:gbm2nd}
%    b_N(x)
%    =
%    \argmin_{b \in \AA}
%        \sum_{i = 1}^{\ell}
%            \left(
%                b(x_i) - \frac{s_i}{h_i}
%            \right)^2,
%\end{equation}
%где через~$h_i$ мы обозначили вторые производные функции потерь:
%\[
%    h_i
%    =
%    \left.
%    \frac{\partial^2 L}{\partial z^2}
%    \right|_{z = a_{N - 1}(x_i)}
%\]

\subsection{Альтернативный подход}
На прошлой лекции мы аргументировали использование среднеквадратичной функции потерь тем,
что она наиболее проста для оптимизации.
Попробуем найти более адекватное обоснование этому выбору.

Мы хотим найти алгоритм~$b(x)$, решающий следующую задачу:
\[
    \sum_{i = 1}^{\ell}
        L(y_i, a_{N - 1}(x_i) + b(x_i))
    \to
    \min_{b}
\]
Разложим функцию~$L$ в каждом слагаемом в ряд Тейлора до второго члена
с центром в ответе композиции~$a_{N - 1}(x_i)$:
\begin{align*}
    \sum_{i = 1}^{\ell}&
        L(y_i, a_{N - 1}(x_i) + b(x_i))
    \approx\\
    &\approx
    \sum_{i = 1}^{\ell} \left(
        L(y_i, a_{N - 1}(x_i))
        -
        s_i b(x_i)
        +
        \frac12
        h_i b^2(x_i)
    \right),
\end{align*}
где через~$h_i$ обозначены вторые производные по сдвигам:
\[
    h_i
    =
    \left.
    \frac{\partial^2}{\partial z^2}
    L(y_i, z)
    \right|_{a_{N - 1}(x_i)}
\]
Первое слагаемое не зависит от нового базового алгоритма, и поэтому его можно выкинуть.
Получаем функционал
\begin{equation}\label{eq:gbmProblem}
    \sum_{i = 1}^{\ell} \left(
        -
        s_i b(x_i)
        +
        \frac12
        h_i b^2(x_i)
    \right)
    \to
    \min_{b}
\end{equation}

Покажем, что он очень похож на среднеквадратичный из формулы~\eqref{eq:gbm}.
Преобразуем его:
\begin{align*}
    \sum_{i = 1}^{\ell}&
        \left(
            b(x_i) - s_i
        \right)^2\\
    &=
    \sum_{i = 1}^{\ell} \left(
        b^2(x_i)
        -
        2 s_i b(x_i)
        +
        s_i^2
    \right) = \{\text{последнее слагаемое не зависит от $b$}\} \\
    &=
    \sum_{i = 1}^{\ell} \left(
        b^2(x_i)
        -
        2 s_i b(x_i)
    \right)\\
    &=
    2
    \sum_{i = 1}^{\ell}
    \left(
        -
        s_i
        b(x_i)
        +
        \frac12
        b^2(x_i)
    \right)
    \to
    \min_{b}
\end{align*}
Видно, что последняя формула совпадает с~\eqref{eq:gbmProblem}
с точностью до константы, если положить~$h_i = 1$.
Таким образом, в обычном градиентном бустинге мы используем
аппроксимацию второго порядка при обучении очередного базового алгоритма,
и при этом отбрасываем информацию о вторых производных (то есть считаем,
что функция имеет одинаковую кривизну по всем направлениям).
%Можно считать, что в~\eqref{eq:gbmProblem} они отброшены
%для большей устойчивости функционала~--- из-за них может произойти деление
%на ноль в окрестности оптимума.

\subsection{Регуляризация}
Будем далее работать с функционалом~\eqref{eq:gbmProblem}.
Он измеряет лишь ошибку композиции после добавления нового алгоритма,
никак при этом не штрафуя за излишнюю сложность этого алгоритма.
Ранее мы решали проблему переобучения путем ограничения глубины деревьев,
но можно подойти к вопросу и более гибко.
Мы выясняли, что дерево~$b(x)$ можно описать формулой
\[
    b(x)
    =
    \sum_{j = 1}^{J}
        b_{j}
        [x \in R_{j}]
\]
Его сложность зависит от двух показателей:
\begin{enumerate}
    \item Число листьев~$J$. Чем больше листьев имеет дерево, тем сложнее его разделяющая поверхность,
        тем больше у него параметров и тем выше риск переобучения.
    \item Норма коэффициентов в листьях~$\|b\|_2^2 = \sum_{j = 1}^{J} b_j^2$.
        Чем сильнее коэффициенты отличаются от нуля, тем сильнее данный
        базовый алгоритм будет влиять на итоговый ответ композиции.
\end{enumerate}
Добавляя регуляризаторы, штрафующие за оба этих вида сложности, получаем следующую задачу:
\[
    \sum_{i = 1}^{\ell} \left(
        -
        s_i b(x_i)
        +
        \frac12
        h_i b^2(x_i)
    \right)
    +
    \gamma J
    +
    \frac{\lambda}{2}
    \sum_{j = 1}^{J}
        b_j^2
    \to
    \min_{b}
\]
Если вспомнить, что дерево~$b(x)$ дает одинаковые ответы на объектах,
попадающих в один лист, то можно упростить функционал:
\[
    \sum_{j = 1}^{J} \Biggl\{
        \underbrace{
        \Biggl(
            -
            \sum_{i \in R_j} s_i
        \Biggr)
        }_{=-S_j}
        b_j
        +
        \frac12
        \Biggl(
            \lambda
            +
            \underbrace{
            \sum_{i \in R_j} h_i
            }_{=H_j}
        \Biggr)
        b_j^2
        +
        \gamma
    \Biggr\}
    \to
    \min_{b}
\]
Каждое слагаемое здесь можно минимизировать по~$b_j$ независимо.
Заметим, что отдельное слагаемое представляет собой параболу относительно~$b_j$,
благодаря чему можно аналитически найти оптимальные коэффициенты в листьях:
\[
    b_j
    =
    \frac{S_j}{H_j + \lambda}
\]
Подставляя данное выражение обратно в функционал, получаем,
что ошибка дерева с оптимальными коэффициентами в листьях вычисляется по формуле
\begin{equation}
\label{eq:impurity}
    H(b)
    =
    -
    \frac12
    \sum_{j = 1}^{J}
        \frac{
            S_j^2
        }{
            H_j + \lambda
        }
    +
    \gamma J
\end{equation}

\subsection{Обучение решающего дерева}
Мы получили функционал~$H(b)$, который для заданной структуры дерева
вычисляет минимальное значение ошибки~\eqref{eq:gbmProblem}, которую можно получить путем подбора коэффициентов в листьях.
Заметим, что он прекрасно подходит на роль критерия информативности~---
с его помощью можно принимать решение, какое разбиение вершины является наилучшим!
Значит, с его помощью мы можем строить дерево.
Будем выбирать разбиение~$[x_j < t]$ в вершине~$R$ так, чтобы оно решало следующую задачу максимизации:
\[
    Q = H(R) - H(R_\ell) - H(R_r) \to \max,
\]
где информативность вычисляется по формуле
\[
    H(R)
    =
    -
    \frac12
    \left(
        \sum_{(h_i, s_i) \in R} s_j
    \right)^2
    \Bigg/
    \left(
        \sum_{(h_i, s_i) \in R} h_j
        +
        \lambda
    \right)
    +
    \gamma.
\]
За счет этого мы будем выбирать структуру дерева так, чтобы оно как можно лучше
решало задачу минимизации исходной функции потерь.
При этом можно ввести вполне логичный критерий останова: вершину нужно объявить листом,
если даже лучшее из разбиений приводит к отрицательному значению функционала~$Q$.

\subsection{Заключение}
Итак, градиентный бустинг в XGBoost имеет ряд важных особенностей.
\begin{enumerate}
    \item Базовый алгоритм приближает направление, посчитанное с учетом вторых производных функции потерь.
    %\item Отклонение направления, построенного базовым алгоритмом,
    %    измеряется с помощью модифицированного функционала~--- из него удалено деление на вторую производную,
    %    за счет чего избегаются численные проблемы.
    \item Функционал регуляризуется~--- добавляются штрафы за количество листьев и за норму коэффициентов.
    \item При построении дерева используется критерий информативности, зависящий от оптимального
        вектора сдвига.
    \item Критерий останова при обучении дерева также зависит от оптимального сдвига.
\end{enumerate}

\section{Стекинг}

Разумеется, существуют способы построения композиций помимо бустинга и бэггинга.
Большую популярность имеет~\emph{стекинг}, в котором прогнозы алгоритмов объявляются
новыми признаками, и поверх них обучается ещё один алгоритм~(который иногда называют мета-алгоритмом).
Стекинг очень популярен в соревнованиях по анализу данных, поскольку позволяет агрегировать
разные модели (различные композиции, линейные модели, нейросети и т.д.; иногда в качестве
базовых алгоритмов могут выступать результаты градиентного бустинга с разными
значениями гиперпараметров).

Допустим, мы независимо обучили~$N$ базовых алгоритмов~$b_1(x), \dots, b_N(x)$ на выборке~$X$,
и теперь хотим обучить на их прогнозах мета-алгоритм~$a(x)$.
Самым простым вариантом будет обучить его на этой же выборке:
\[
    \sum_{i = 1}^{\ell}
        L(y_i, a(b_1(x_i), \dots, b_N(x_i)))
    \to
    \min_{a}
\]
При таком подходе~$a(x)$ будет отдавать предпочтение тем базовым алгоритмам,
которые сильнее всех подогнались под целевую переменную на обучении~(поскольку по их прогнозам
лучше всего восстанавливаются истинные ответы).
Если среди базовых алгоритмов будет идеально переобученный~(то есть запомнивший
ответы на всей обучающей выборке), то мета-алгоритму будет выгодно
использовать только прогнозы данного переобученного базового алгоритма,
поскольку это позволит добиться лучших результатов с точки зрения записанного функционала.
При этом такой мета-алгоритм, конечно, будет показывать очень низкое качество на новых данных.

Чтобы избежать таких проблем, следует обучать базовые алгоритмы и мета-алгоритм
на разных выборках.
Разобьём нашу обучающую выборку на~$K$ блоков~$X_1, \dots, X_K$,
и обозначим через~$b_j^{-k}(x)$ базовый алгоритм~$b_j(x)$, обученный по всем блокам,
кроме~$k$-го.
Тогда функционал для обучения мета-алгоритма можно записать как
\[
    \sum_{k = 1}^{K}
    \sum_{(x_i, y_i) \in X_k}
        L\left(
            y_i,
            a(b_1^{-k}(x_i), \dots, b_N^{-k}(x_i))
        \right)
    \to
    \min_{a}
\]
В данном случае при вычислении ошибки мета-алгоритма на объекте~$x_i$
используются базовые алгоритмы, которые не видели этот объект при обучении,
и поэтому мета-алгоритм не может переобучиться на их прогнозах.

\paragraph{Блендинг.}
Частным случаем стекинга является блендинг, в котором мета-алгоритм является линейным:
\[
    a(x)
    =
    \sum_{n = 1}^{N}
        w_n b_n(x).
\]
Это самый простой способ объединить несколько алгоритмов в композицию.
Иногда даже блендинг без обучения весов~(то есть вариант с~$w_1 = \dots = w_N = 1/N$)
позволяет улучшить качество по сравнению с отдельными базовыми алгоритмами.

\paragraph{Категориальные и текстовые признаки.}
Категориальные и текстовые признаки могут быть серьёзной помехой для использования композиций над деревьями.
Стандартным способом кодирования является бинаризация для категориальных признаков
и TF-IDF для текстовых, что приводит к очень большой размерности признакового пространства.
Случайный лес на таком наборе признаков будет обучаться долго из-за большой глубины деревьев,
а градиентный бустинг может показать слишком плохие результаты из-за небольшой
глубины базовых деревьев~(например, глубина 4 позволяет учитывать лишь зависимость целевой переменной
от наборов из 4-х признаков; в случае с текстами ответы могут зависеть от существенно более крупных наборов слов).

Одним из решений этой проблемы может стать стекинг, в котором градиентным бустингом обучается мета-алгоритм~$a(x)$,
а каждый категориальный и текстовый признак схлопывается в одно число соответствующим базовым алгоритмом.
Для категориальных признаков базовый алгоритм, например, может вычислять счётчики~---
при этом обратим внимание, что мы уже отмечали важность разделения обучающих выборок
для счётчиков и настраиваемых поверх них моделей.
Также популярным выбором для базовых алгоритмов являются линейные модели.


\begin{thebibliography}{1}
\bibitem{chen16xgboost}
    \emph{Tianqi Chen, Carlos Guestrin} (2016).
    XGBoost: A Scalable Tree Boosting System.~//
    \url{http://arxiv.org/abs/1603.02754}
\end{thebibliography}
\end{document}
